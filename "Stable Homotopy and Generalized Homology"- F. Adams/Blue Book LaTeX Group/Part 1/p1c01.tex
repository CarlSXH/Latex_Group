\documentclass[../main]{subfiles}
\begin{document}
\label{sec:p1c1}

\chapter{Introduction}

The work of S.P. Novikov which is in question was presented at the International Congress of Mathematics, Moscow, 1966, in a half-hour lecture, in a seminar and in private conversations. It has also been announced in the Doklady of the Academy of Sciences of the USSR, vol. 172 (1967) pp.~33--36. Some of Novikov's results have been obtained independently by F.S. Landweber (to appear in the Transactions of the AMS).

The object of these seminar notes is to give an exposition of that part of Novikov's work which deals with operations of complex cobordism. I hope that this will be useful, because I believe that the cohomology functor provided by complex cobordism is now ripe for exploitation. I therefore aim to present the material in sufficient detail, so that a reader who has a concrete application in mind can make his own calculations. In particular, I will give certain formulae which are not made explicit in the sources cited above.

These notes will not deal with any of the other topics which are mentioned in the sources cited above. These include the following. \\[6pt]
\begin{enumerate}[label=(\roman*)] % we're using enumitem it seems --deri
\item Generalizations of the Adams spectral sequence in which ordinary cohomology is replaced by generalized (extraordinary) cohomology.
\item Connections between these studies for complex cobordism $\Omega_U^* (X, Y)$ and the corresponding studies for complete K-theory $K^* (X, Y)$
\item The cohomology functor $\Omega_U^* (X, Y) \bigotimes \bbQ_p$ (where $\bbQ_p$ is the ring of rational numbers $a/b$ with $b$ prime to $p$); and the splitting of this functor into direct summands. %Q_p -> \bbQ_p --George
\end{enumerate}


\end{document}