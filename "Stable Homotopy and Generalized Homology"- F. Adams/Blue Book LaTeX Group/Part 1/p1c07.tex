\documentclass[../main]{subfiles}
\begin{document}
\label{sec:p1c7}

% Manan
% The citations might need review

\chapter{Scholium on Novikov's Exposition}

In Moscow, Novikov made a careful distinction, which is maintained in his Doklady note, between $s_{\omega}\colon\Omega_U^{\ast}(P)\longrightarrow\Omega_U^{\ast}(P)$ and a certain homomorphism $\sigma_{\omega}^{\ast}\colon\Omega_U^{\ast}(P)\longrightarrow\Omega_U^{\ast}(P)$. It is necessary to observe that they coincide, and for this purpose it is necessary to analyse \plscite{Theorem 3 of Novikov's Doklady note}. %does this warrant a citation?

First observe that in Novikov's Doklady note, $M_U$ and $\Omega_U$ are different names for the same thing, since both are defined to be $\Omega_U^{\ast}(P)$ \plscite{(p. 33 line 4 of Section II; p. 35 line 8)}. Next recall that Novikov writes $A^U$ for the algebra of operations, and observe that the isomorphism 
\begin{center}
    \begin{tikzcd}
    \te{Hom}_{A^U}(A^U,M_U)\arrow[r,"\cong"]&\Omega_U
    \end{tikzcd}
\end{center}
which he has in mind is precisely the standard isomorphism $\theta$ given by
\begin{equation*}
    \theta(h)=h(1).
\end{equation*}
Next consider Novikov's map $d\colon A^U\longrightarrow A^U$. Since it is asserted to induce a map
\begin{center}
    \begin{tikzcd}
    d^{\ast}\colon\te{Hom}_{A^U}(A^U,M_U)\arrow[r]&\te{Hom}_{A^U}(A^U,M_U),
    \end{tikzcd}
\end{center}
it is implicit that $d$ must be a map of left $A$-modules. Since it is asserted to satisfy $d(1)=s_{\omega}$, it must be given by
\begin{equation*}
    d(a)=as_{\omega}.
\end{equation*}

Now consider the following diagram.
\begin{center}
    \begin{tikzcd}[row sep=large,column sep=large]
    \te{Hom}_{A^U}(A^U,M_U)\arrow[d,"\theta"]\arrow[r,"d^{\ast}"]&\te{Hom}_{A^U}(A^U,M_U)\arrow[d,"\theta"]\\
    \Omega_U\arrow[r,"x"]&\Omega_U
    \end{tikzcd}
\end{center}

It is trivial to check that it is commutative if we define $x$ by $x(y)=s_{\omega}y$. But Novikov asserts that it is commutative if define $x$ to be $\sigma_{\omega}^{\ast}$. Therefore $\sigma_{\omega}^{\ast}(y)=s_{\omega}y$.
\end{document}