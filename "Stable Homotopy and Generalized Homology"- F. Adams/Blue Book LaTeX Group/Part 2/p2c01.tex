\documentclass[../main]{subfiles}
\begin{document}
\label{sec:p2c1}

\chapter{Formal Groups}
We may understand formal groups by an analogy. Let $G$ be a real Lie group of dimension $1$. By choosing a chart, we may identify a neighbourhood of the unit in $G$ with a neighbourhood of zero in ${\bbR}^1$, so that the unit of $G$ corresponds to zero. The product in $G$ is then given by a power-series:

\begin{equation}
\tag{1.1}
\label{eqn:p2c01.1}
\mu(x, y) = \sum_{i, j \ge 0} a_{ij} x^i y^j.
\end{equation}

This power-series is convergent for small $x$ and $y$ and satisfies the following conditions.

\begin{equation}
\tag{1.2}
\label{eqn:p2c01.2}
\mu(x,0) = x, \quad \mu(0, y) = y.
\end{equation}
\begin{equation}
\tag{1.3}
\label{eqn:p2c01.3}
\mu(x, \mu(y, z)) = \mu(\mu(x, y), z).
\end{equation}

Now let $R$ be any commutative ring with unit. Then a ``formal product'' (over $R$) is a formal power series of the form \eqref{eqn:p2c01.1}, but with coefficients $a_{ij}$ in $R$, satisfying \eqref{eqn:p2c01.2} and \eqref{eqn:p2c01.3}. 

We have two trivial examples.

\begin{equation}
\tag{1.4} 
\label{eqn:p2c01.4}
\mu(x, y) = x + y,
\end{equation}
\begin{equation}
\tag{1.5}
\label{eqn:p2c01.5}
\mu(x, y) = x + y + xy.
\end{equation}

For example, suppose that we consider the Lie groups $G$ of positive real numbers under multiplication, and use the chart under which $x \in {\bbR}^1$ corresponds to $(1 + x) \in G$; we obtain formula \eqref{eqn:p2c01.5}.

Let us return to the general case; there a few obvious comments. Condition \eqref{eqn:p2c01.2} is equivalent to 

\begin{equation}
\tag{1.6}
\begin{split}
a_{i0} & = \begin{cases}1 & i = 1 \\ 0 & i \ne 1\end{cases} \\
a_{0j} & = \begin{cases}1 & j = 1 \\ 0 & j \ne 1\end{cases}
\end{split}
\end{equation}

So we may write our formal power-series in the following form 

\begin{equation}
\tag{1.7}
\label{eqn:p2c01.7}
\mu(x, y) = x + y + \sum_{i, j \ge 0} a_{ij} x^i y^j.
\end{equation}

Condition \eqref{eqn:p2c01.3} involves substituting one formal power-series into another, but this involves no difficulty since our formal power-series have their constant terms zero.

We observe that so far we are only discussing the case of dimension $1$. That is, in the general case one would start from a Lie group of dimension $n$, and proceed by analogy.

Given a formal product $\mu$, a \emph{formal inverse} $\iota$ is a formal power-series
\begin{equation}
\tag{1.8}
\label{eqn:p2c01.8}
\iota x = \sum_{j \ge 1} a_j' x^j
\end{equation}

(with coefficients $a_j'$ in our ring $R$) such that 
\begin{equation}
\tag{1.9}
\label{eqn:p2c01.9}
\mu(x, \iota x) = 0, \quad \mu(\iota x, x) = 0
\end{equation}

\begin{lemma}
\label{lem:p2c01.10}
Given any formal product $\mu$, there is a formal inverse $\iota$, and it is unique.
\end{lemma}

The proof is trivial. 

We have two examples; with the ``additive product'' of \eqref{eqn:p2c01.4} we have $$\iota(x) = -x,$$ and with the ``multiplicative product'' of \eqref{eqn:p2c01.5} we have $$\iota(x) = -x + x^2 - x^3 + x^4 \ldots.$$ So far, a ``formal product'' is like a grin without a Cheshire cat behind it. A ``formal group'' must, of course, be a group object in a suitable category; I take this notion as known. If $X$ is to be a group object in the category $C$, then Cartesian products such as $X^n$ must exist in $C$ for $n = 0, 1, 2, 3$; and $X$ must be provided with structure maps in the category $C$, namely a product map $m : X^2 \lar{} X$, a unit map $e : X^0 \lar{} X$ and an inverse map $i : X \lar{} X$. These maps must satisfy the obvious conditions. For example, consider the category of smooth manifolds and smooth maps; a group in this category is a Lie group. Again, consider the category of commutative maps and homomorphisms of rings, and let $C$ be the opposite category; with a little goodwill $C$ may be regarded as the category of affine algebraic varieties. A group in this category is an ``algebraic group''.\\

Now consider the category in which the objects are filtered commutative algebras over $R$, which are complete and Hausdorff for the filtration topology; the morphisms are filtration-preserving homomorphisms. Let $C$ be the opposite category. The ring of formal power-series $$R[\![x_1, x_2, \ldots, x_n]\!],$$ with the obvious filtration, is an object in $C$. The objects $R[\![x_1, x_2, \ldots, x_n]\!]$ and $R[\![y_1, y_2, \ldots, y_m]\!]$ have a Cartesian product in $C$, namely $$R[\![x_1, x_2, \ldots, x_n, y_1, y_2, \ldots, y_m]\!].$$ Let $X$ be the object $R[\![x]\!]$ in $C$, then a map $m : X^2 \lar{} X$ in $C$ is a filtration-preserving homomorphism $$m : R[\![x]\!] \lar{} R[\![x_1, x_2]\!];$$ such a map $m$ is determined by giving $m(x)$, which is a formal power-series $\mu(x_1, x_2)$ with zero constant term. It is now easy to check that each ``formal product'' $\mu$ determines a structure map $m$ which makes $R[\![x]\!]$ into a group object, and conversely. (The unit map $e : R[\![x]\!] \lar{} R$ defined by $\displaystyle e \left(\sum_{i \ge 0} c_i x^i\right) = c_0$; inverse  maps come free of charge by Lemma~\ref{lem:p2c01.10}). It is now clear how to proceed in dimension $n$; we have to consider the object $R[\![x_1, x_2, \ldots, x_n]\!]$, and study the ways of making it into a group-object in $C$. A ``formal group'', then, is a group-object in the category $C$, whose underlying object is $R[\![x_1, x_2, \ldots, x_n]\!]$.

We now revert to the case of dimension $1$. Let $\theta : R \lar{} S$ be a homomorphism of rings with unit. Then $\theta$ induces the map $$\theta_\ast : R[\![x_1, x_2]\!] \lar{} S[\![x_1, x_2]\!]$$ which carries any formal product $\mu$ over $R$ into a formal product $\theta_\ast \mu$ over $S$. However, this is not the definition of a homomorphism between formal groups. Such a homomorphism is, of course, a map in our category, with the obvious property. That is, if $G$ is a formal group $(R[\![x]\!], \mu)$ and $H$ is a formal group $(R[\![y]\!], \nu)$, then a homomorphism $\theta : G \lar{} H$ is a formal power series $$y = f(x) = \sum_{i \ge 1} c_i x^i$$ (with coefficients $c_i$ in $R$) such that $$\nu(f(x_1), f(x_2)) = f\mu(x_1, x_2).$$ The analogy with the case of a Lie group is obvious. If the coefficient $c_1$ is invertible in $R$, then $f^{-1}$ exists, and $f$ is an isomorphism.

In our applications we are interested only in the case of dimension $1$, and moreover only in commutative formal groups. That is, our formal products will satisfy 
\begin{equation}
\tag{1.11}
\label{eqn:p2c01.11}
\mu(x, y) = \mu(y, x),
\end{equation}

or equivalently
\begin{equation}
\tag{1.12}
\label{eqn:p2c01.12}
a_{ij} = a_{ji}.
\end{equation}

Our applications arise in algebraic topology.
\end{document}