\documentclass[../main]{subfiles}
\begin{document}
\label{sec:p2c4}
\renewcommand{\labelenumi}{(\roman{enumi})}

\chapter{Calculations in $E$-Homology and Cohomology}

In this section we continue the programme of taking results which are familiar for ordinary homology and cohomology, and carrying them over to $E$. First we compute the $E$-homology of the spaces $BU(n)$ and $BU$. The space $BU$ is an $H$-space; its product corresponds to addition in $K$-cohomology; in particular, we have the following homotopy-commutative diagram, in which the upper arrow is the Whitney sum map.
~\\~\\
\adjustbox{scale=1.1,center}{
\begin{tikzcd}
BU(n)\times BU(m) &&& BU(n+m) \\
\\
BU\times BU &&& BU
\arrow[from=1-1, to=1-4]
\arrow[from=1-1,to=3-1]
\arrow[from=3-1, to=3-4]
\arrow[from=1-4, to=3-4]
\end{tikzcd}
}
~\\~\\
This diagram gives rise to the following diagram of products.
~\\~\\
\adjustbox{scale=0.95,center}{
\begin{tikzcd}
E_\ast(BU(n))\otimes_{\pi_\ast(E)} E_\ast(BU(m)) &&& E_\ast(BU(n+m)) \\
\\
E_\ast(BU)\otimes_{\pi_\ast(E)} E_\ast(BU) &&& E_\ast(BU)
\arrow[from=1-1, to=1-4]
\arrow[from=1-1,to=3-1]
\arrow[from=3-1, to=3-4]
\arrow[from=1-4, to=3-4]
\end{tikzcd}
}
~\\~\\

By using the injection $BU(1)\longrightarrow BU$, the classes $\beta_i\in E_\ast(\mathbb{CP}^\infty)$ give classes in $E_\ast(BU)$; we write $\beta_i$ for these classes also. The element $\beta_0$ acts as a unit for the products.
\begin{lemma}
\label{lem:p2c04.1}
\begin{enumerate}
    \item The spectral sequences
    $$H_\ast(BU(n);\pi_\ast(E))\Longrightarrow E_\ast(BU(n))$$
    $$H_\ast(BU;\pi_\ast(E))\Longrightarrow E_\ast(BU)$$
    are trivial.
    \item $E_\ast(BU(n))$ is free over $\pi_\ast(E)$, with a base consisting of the monomials
    $$\beta_{i_1}\beta_{i_2}...\beta_{i_r}$$
    such that $i_1>0, i_2>0, ..., i_r>0$, $0\leq r \leq n$. (The monomial with $r=0$ is interpreted as $1$).
    
    $E_\ast(BU)$ is a polynomial algebra
    $$\pi_\ast(E)[\beta_1,\beta_2,...,\beta_i, ...].$$
    \item The coproduct in $E_\ast(BU(n))$ and $E_\ast(BU)$ is given by 
    $$\psi\beta_k=\sum_{i+j=k}\beta_i\otimes\beta_j,$$
    where $\beta_0=1$.
\end{enumerate}
\end{lemma}

The proof of parts (i) and (ii) is easy, because the monomials
$$\beta_{i_1}\beta_{i_2}...\beta_{i_r}$$
give a $\pi_\ast(E)$-base for the $E^2$-term on which all differentials $d_r$ vanish. Since the differentials are linear over $\pi_\ast(E)$, they vanish on everything. Part (iii) comes from \eqref{cor:p2c02.18}.

We now introduce a general lemma.
\begin{lemma}
\label{lem:p2c04.2}
Let $X$ be a space (or a spectrum provided that $\pi_r(X)=0$ for $r<-N$, some N). Suppose that $H_\ast(X;\pi_\ast(E))$ is free over $\pi_\ast(E)$ and that the spectral sequence $H_\ast(X;\pi_\ast(E))\Longrightarrow E_\ast(X)$ is trivial. Let $F$ be a module-spectrum over the ring-spectrum $E$. Then the spectral sequences
$$H_\ast(X;\pi_\ast(F))\Longrightarrow F_\ast(X)$$
$$H^\ast(X;\pi_\ast(F))\Longrightarrow F^\ast(X)$$
are trivial, and the maps
$$E_\ast(X)\otimes_{\pi_\ast(E)}\pi_\ast(F)\longrightarrow F_\ast(X)$$
$$F^\ast(X)\longrightarrow \operatorname{Hom}_{\pi_\ast(E)}(E_\ast(X),\pi_\ast(F))$$
are isomorphisms.
\end{lemma}

The proof is a routine exercise on pairings and spectral sequences (compare \cite[p.~20, Proposition 17]{adams3}).

In particular, if $E$ is as in \hyperref[sec:p2c2]{\S 2}, the lemma applies to $X=\mathbb{CP}^\infty$, $BU(n)$ and $BU$. We will also see that it applies to $X=MU$ -- see \eqref{lem:p2c04.5}.

Although it is quite unnecessary for our main purposes, we pause to observe that Chern classes behave as expected in $E$-cohomology.

\begin{lemma}
\label{lem:p2c04.3}
\begin{enumerate}
    \item $E^\ast(BU)$ contains a unique element $c_i$ such that 
    $$\left<c_i,(\beta_1)^i\right>=1$$
    and 
    $$\left<c_i,m\right>=0$$
    where $m$ is any monomial $\beta_1^{i_1}\beta_2^{i_2}...\beta_r^{i_r}$ distinct from $(\beta_1)^i$. We have $c_0=1$.
    \item The restriction of $c_1$ to $BU(1)$ is $x^E$, the generator given in \hyperref[sec:p2c2]{\S 2}.
    \item The restriction of $c_i$ to $BU(n)$ is zero for $i>n$. (Otherwise, the image of $c_i$ in $E^\ast(BU(n))$ will also be written $c_i$.)
    \item $E^\ast(BU(n))$ is the ring of formal power-series
    $$\pi_\ast(E)[[c_1,c_2,...,c_n]];$$
    and $E^\ast(BU)$ is the ring of formal power-series
    $$\pi_\ast(E)[[c_1,c_2,...,c_i,...]].$$
    \item We have
    $$\psi c_k=\sum_{i+j=k}c_i\otimes c_j.$$
\end{enumerate}
\end{lemma}
\begin{proof}
The definition of $c_i$ in (i) is legitimate by \eqref{lem:p2c04.2} applied to $X=BU$, $F=E$. We easily check that the unit $1\in E_\ast(BU)$ plays the role laid down for $c_0$. Part (ii) is trivial; part (iii) follows easily from \eqref{lem:p2c04.1}(ii) plus \eqref{lem:p2c04.2} applied to $X=BU(n)$. We turn to part (iv).

Let $m$ be a monomial
$$m=\beta_1^{i_1}\beta_2^{i_2}...\beta_r^{i_r} \text{ in } E_\ast(BU);$$
let the image of $m$ under the iterated diagonal, which is determined by \eqref{lem:p2c04.1}(iii), be 
$$\sum_\alpha m_{1\alpha}\otimes m_{2\alpha} \otimes ... \otimes m_{s\alpha}.$$
Then 
$$\left<c_{j_1}c_{j_2}...c_{j_s},m\right> = \sum_\alpha\left<c_{j_1},m_{1\alpha}\right>\left<c_{j_2},m_{2\alpha}\right>...\left<c_{j_s},m_{s\alpha}\right>;$$
and this is a well-determined integer independent of the spectrum $E$. In particular this integer is the same as in the case $E=H$. We conclude that in the spectral sequence
$$H^\ast(BU(n);\pi_\ast(E))\Longrightarrow E^\ast(BU(n)), \text{ or}$$
$$H^\ast(BU;\pi_\ast(E))\Longrightarrow E^\ast(BU)$$
the $E_2$ term has a $\pi_\ast(E)$-base consisting of the appropriate monomials 
$$c_{j_1}c_{j_2}...c_{j_s}.$$
This leads to part (iv). Part (v) follows by duality from the definition in part (i).
\end{proof}

The classes $c_i$ are of course the generalized Chern classes in $E$-cohomology. If required they may be constructed as characteristic classes for $U(n)$-bundles over appropriate spaces by the method of Grothendieck, and then pulled back to $BU(n)$ and $BU$ by limiting arguments. (Compare \cite[pp.~8-9]{adams2}). In the case $E=MU$ we get the Conner-Floyd Chern classes.

If we have more than one spectrum in sight we write $\beta_i^E,c_i^E$. If we are given a map $f:E\longrightarrow F$ of ring-spectra, and choose $x^F=f_\ast x^E$, as in \hyperref[sec:p2c2]{\S 2}, then we have 
$$c_i^F=f_\ast c_i^E.$$
The reader may carry over \eqref{lem:p2c02.15} to cohomology, but it is not necessary for our purposes.

For the next lemma, we note that $E_\ast(MU)$ is a ring, and that the "inclusion" of $MU(1)$ in $MU$ induces a homomorphism
$$\tilde{E}_p(MU(1))\longrightarrow E_{p-2}(MU).$$
Following the analogy of ordinary homology, we take the element
$$u^E\beta_{i+1}^E\in \tilde{E}_\ast(MU(1)) \quad (i\geq 0)$$
and write $b_i^E$ for its image in $E_\ast(MU)$. The factor $u^E$ (see \hyperref[sec:p2c2]{\S 2}) is introduced in order to ensure that $b_0^E=1$ in $E_0(MU)$.

Suppose given a map $f:E\longrightarrow F$ of ring-spectra. Then it is clear that Lemma~\ref{lem:p2c02.15} carries over; with the notation \eqref{lem:p2c02.15}, we have the following result.
\begin{equation}
\label{eqn:p2c04.4}
\tag{4.4}
    f_\ast b_i^E=c_1\sum_jd_{i+1}^{j+1}b_j^F.
\end{equation}
In particular, as soon as we obtain the canonical map $f:MU\longrightarrow H$, it will send $b_i^{MU}$ to $b_i^H$; as soon as we obtain the canonical map $g:MU\longrightarrow K$, it will send $b_i^{MU}$ to $u^ib_i^K$, where $u=u^K\in \pi_2(K)$.

With an eye to later applications (\hyperref[sec:p2c15]{\S 15}) we include a little spare generality in the next two lemmas. Let $R$ be a subring of the rational numbers $\mathbb{Q}$; the reader interested only in the immediate applications may take $R=\mathbb{Z}$. We recall from \hyperref[sec:p2c2]{\S 2} that $MUR$ is the representing spectrum for complex bordism and cobordism with coefficients in $R$.

We assume that for each integer $d$ invertible in $R$, the groups $\pi_\ast(E)$ have no $d$-torsion. This assumption is certainly vacuous if $R=\mathbb{Z}$.

\begin{lemma}
\label{lem:p2c04.5}
\begin{enumerate}
    \item The spectral sequences
    $$H_\ast(MUR;\pi_\ast(E))\Longrightarrow E_\ast(MUR)$$
    $$H_\ast(MUR\wedge MUR; \pi_ast(E))\Longrightarrow E_\ast(MUR\wedge MUR)$$
    are trivial, so that Lemma \ref{lem:p2c04.2} applies.
    \item $E_\ast(MUR)$ is the polynomial ring 
    $$(\pi_\ast(E)\otimes R)[b_1,b_2,...,b_n,...].$$
\end{enumerate}
\end{lemma}
\begin{proof}
For (i), in the case of $MUR$ we note that the monomials in the $b_i$ form a $\pi_\ast(E)\otimes R$-base for the $E_2$-term on which all differentials $d_r$ vanish. The differentials $d_r$ are linear over $\pi_\ast(E)$, and by using the assumption on $\pi_\ast(E)$ we see they are linear over $R$. So the differentials $d_r$ vanish on everything. Similarly for $MUR\wedge MUR$, using exterior products of such monomials. This proves (i) and (ii).
\end{proof}

For the next lemma, let $R$ be again a subring of the rational numbers $\mathbb{Q}$, and let $E$ be a ring-spectrum, with $x^E$ as in \hyperref[sec:p2c2]{\S 2}, such that 
$$\pi_\ast(E)\longrightarrow\pi_\ast(E)\otimes R$$
is iso. (For example we might have $E=FR$)

\begin{lemma}
\label{lem:p2c04.6}
Suppose given a formal power-series
$$f(x^E)=\sum_{i\geq 0}d_i(x^E)^{i+1}\in\tilde{E}^2(\mathbb{CP}^\infty)$$
with $u^Ed_0=1$. Then there is one and (up to homotopy) only one map of ring-spectra
$$g:MUR\longrightarrow E$$
such that $g_\ast x^{MU}=f(x^E)$.
\end{lemma}
\begin{notes}
~
\begin{enumerate}
    \item By abuse of language, we have written $x^MU$ also for the image of $x^MU\in\widetilde{MU}^2(\mathbb{CP}^\infty)$ in $\widetilde{MUR}^2(\mathbb{CP}^\infty)$
    \item The necessity of the condition $u^Ed_0=1$ is shown by \ref{eqn:p2c02.12}.
\end{enumerate}
\end{notes}
\begin{proof}
We check that the conditions of Lemma \ref{lem:p2c04.2} apply to $X=MUR$, $F=E$. We certainly have
$$H_\ast(MUR;\pi_\ast(E))\cong H_\ast(MU;\pi_\ast(E)\otimes R)\cong H_\ast(MU;\pi_\ast(E))$$
(by the assumption on $E$), so $H_\ast(MUR;\pi_\ast(E))$ is free over $\pi_\ast(E)$. 

Similarly
$$E_\ast(MUR)=(\pi_\ast(E)\otimes R)[b_1,b_2,...,b_n,...]=\pi_\ast(E)[b_1,b_2,...,b_n,...].$$
If $\pi_\ast(E)\longrightarrow \pi_\ast(E)\otimes R$ is iso, then $\pi_\ast(E)$ has no $d$-torsion for any integer $d$ invertible in $R$, and Lemma \ref{lem:p2c04.5} shows that the spectral sequence
$$H_\ast(MUR;\pi_\ast(E))\Longrightarrow E_\ast(MUR)$$
is trivial. So Lemma \ref{lem:p2c04.2} shows that there is a $1$-$1$ correspondence between homotopy classes of maps
$$g:MUR\longrightarrow E$$
and maps
$$\theta:E_\ast(MUR)\longrightarrow \pi_\ast(E)$$
which are linear over $\pi_\ast(E)$, and of degree zero. Similarly for maps
$$h:MUR\wedge MUR \longrightarrow E;$$
and this allows us to check whether a map $g:MUR\longrightarrow E$ makes the following diagram homotopy-commutative.
~\\
$$\begin{tikzcd}
MUR\wedge MUR &&& E\wedge E \\
\\
MUR &&& E
\arrow["g\wedge g", from=1-1, to=1-4]
\arrow[from=1-1, to=3-1]
\arrow["g", from=3-1, to=3-4]
\arrow[from=1-4, to=3-4]
\end{tikzcd}$$
~\\
We find by diagram-chasing that for this, it is necessary and sufficient that the map $\theta$ corresponding to $g$ should be a map of algebras over $\pi_\ast(E)$. Now the condition
$$g_\ast x^{MU}=\sum_{i\geq 0} d_i(x^E)^{i+1}$$
is equivalent to
$$\theta(b_i)=u^Ed_i\quad (i\geq 0).$$
Provided that $u^Ed_0=1$, there is one and only one map $\theta$ of $\pi_\ast(E)$-algebras which satisfies this condition. This proves Lemma \ref{lem:p2c04.6}.
\end{proof}
\begin{example}
There is one and only one map $f:MU\longrightarrow H$ of ring-spectra such that 
$$f_\ast x^MU=x^H.$$
This is of course a trivial example.
\end{example}
\begin{example}
There is one and only one map $g:MU\longrightarrow K$ of ring-spectral such that
$$g_\ast x^{MU}=\left(u^K\right)^{-1}x^K.$$
This map is, of course, the usual one, which provides a $K$-orientation for complex bundles.
\end{example}

We can also use Lemma \ref{lem:p2c04.6} to construct multiplicative cohomology operations from $MU^\ast$ to $MU^\ast$, following Novikov \cite{novikov}.

We can also use Lemma \ref{lem:p2c04.6} to obtain Hirzebruch's theory of "multiplicative sequences of polynomials" in the (ordinary) Chern classes. If we think for a moment about the gradings in Hirzebruch's formulae, we see tjhat for this purpose we need to take $E$ to be a product of Eilenberg-Maclane spectra, having homotopy groups
$$\pi_r(E)=\begin{cases}\mathbb{Q} & \text{for } r \text{ even} \\ 0 & \text{for } r \text{ odd}\end{cases}$$
A suitable candidate is the spectrum $H\wedge K$, which has the required properties.

Some readers may perhaps be used to thinking of ``multiplicative sequences of polynomials'' as elements of the cohomology of the space $BU$ (elements of $(H\wedge K)^\ast(BU)$, in fact); and they may pergaps be surprised to see them treated as maps of $MU$. On this point several comments are in order.
\begin{enumerate}
    \item[(a)] Lemma \ref{lem:p2c04.6} provides us with all the Thom classes we need, so we have a Thom isomorphism
    $$(H\wedge K)^\ast(BU)\cong (\wedge K)^\ast(MU).$$
    \item[(b)] "Multiplicative sequences of polynomials" carry the Whitney sum in BU into the product in cohomology. The Whitney sum in BU corresponds to the product in $MU$, so it is more convenient to describe the behavior on products by saying that we have a map of ring-spectra defined in the spectrum $MU$.
    \item[(c)] "Multiplicative sequences of polynomials" are intended for use on manifolds, so that we actually require their values on elements of $\pi_\ast(MU).$ For this reason, their definition in terms of $MU$ may be more transparent than their expression in terms of ordinary Chern classes in $BU$. For example, consider the map of ring-spectra
    $$MU\overset{g}{1\longrightarrow} K\cong S^0\wedge K \longrightarrow H\wedge K,$$
    where the map $g:MU\longrightarrow K$ is that mentioned above.
\end{enumerate}

\begin{exercise}
Follow up these hints.
\end{exercise}

Lemma \ref{lem:p2c04.6} shows that if we consider pairs $(E,x^E),$ as above, and such that $u^E=1$, then among them the pair $(MU,x^{MU})$ has a universal property; for any other pair $(E,x^E)$, there is a map $g:MU\longrightarrow E$ such that $g_\ast x^{MU}=x^E$. In particular, for any such $(E,x^E)$ we have a homomorphism of rings $g_\ast:\pi_\ast(MU)\longrightarrow \pi_\ast(E)$ such that $g_\ast\mu^{MU}=\mu^E$ (see \hyperref[sec:p2c2]{\S 2}); that is, $g_\ast$ carries the one formal product into the other. We will see in the next section that there is a ring $L$, with a formal product defined over it, which enjoys a similar universal property in a purely algebraic setting. It is known that $\pi_\ast(MU)$ is a polynomial algebra, over $\mathbb{Z}$, on generators of dimension $2,4,6,8,...$. The ring $L$ can be made into a graded ring, and it is known that it is then a polynomial algebra, over $\mathbb{Z}$, on generators of dimension $2,4,6,8,...$. Following Quillen, we regard these as plausibility arguments, to introduce the theorem that the canonical map from $L$ to $\pi_\ast(MU)$ is an isomorphism.
\end{document}
