\documentclass[../main]{subfiles}
\begin{document}
\label{sec:p2c6}
\renewcommand{\labelenumi}{(\roman{enumi})}

\chapter{More calculations in $E$-Homology}
The element $a_{ij}$ in $\pi_{2(i+j-1)}(MU)$ can be represented by a weakly almost-complex manifold; we might well be asked to compute the (normal) characteristic numbers of this manifold. It is equivalent to ask for the image of $a_{ij}$ under the Hurewicz homomorphism
$$\pi_\ast(MU)\longrightarrow H_\ast(MU).$$
It is the object of this section to this answer this question.

To do so we introduce the Boardman homomorphism, which is slightly more general than the Hurewicz homomorphism. Let $E$ be a (commutative) ring-spectrum; then for any (space or spectrum) $Y$ we can consider the map 
$$Y\simeq S^0\wedge Y\overset{i\wedge 1}{\longrightarrow} E\wedge Y;$$
composition with this map induces a homomorphism 
$$[X,Y]_\ast\overset{B}{\longrightarrow}[X,E\wedge Y]_\ast.$$
We recover the Hurewicz homomorphism by setting $X=S^0$, $E=H$. 

The Boardman homomorphism is more or less guaranteed to be useful when $E=H$, because of the following lemma.
\begin{lemma}
\label{lem:p2c06.1}
$H\wedge Y$ is equivalent (though not canonically, in general) to a product of Eilenberg-maclane spectra, whose homotopy groups are the groups
$$\pi_n(H\wedge Y)=H_n(Y).$$
It follows that 
$$[X,H\wedge Y]_r \cong \prod_n H^{n-r}(X;H_n(Y))$$
(not canonically); so the groups $[X,E\wedge Y]_r$ are computable for $E=H$.
\end{lemma}
\begin{proof}
For each $n$, we can construct a Moore spectrum $M(G_n,n)$ for the group $G_n=\pi_n(H\wedge Y)$ in dimension $n$, and construct a map 
$$f_n:M(G_n,n)\longrightarrow H\wedge Y$$
which induces an isomorphism
$$(f_n)_\ast:\pi_n(M(G_n,n))\longrightarrow \pi_n(H\wedge Y).$$
We can then construct the map
$$H\wedge M(G_n,n)\overset{1\wedge f_n}{\longrightarrow} H\wedge H\wedge Y \overset{\mu\wedge 1}{\longrightarrow} H\wedge Y,$$
where $H\wedge M(G_n,n)$ is an Eilenberg-Maclane spectrum for the group $G_n$ in dimension $n$. We can then form the map
$$\bigvee_n H\wedge M(G_n,n)\longrightarrow H\wedge Y$$
whose $n$-th component is the map $(\mu\wedge 1)(1\wedge f_n)$ just constructed; we observe that it is a homotopy equivalence by Whitehead's Theorem (in the category of spectra). Let $\prod_n H\wedge M(G_n,n)$ be the product in the categorical sense; then there is a map 
$$\bigvee_n H\wedge M(G_n,n)\longrightarrow \prod_n H\wedge M(G_n,n),$$
and this too is a homotopy equivalence by Whitehead's Theorem. This proves Lemma \ref{lem:p2c06.1}.
\end{proof}

Returning to the general case, since $E\wedge Y$ is at least a module-spectrum over the ring-spectrum $E$, we may hope to obtain information about $[X,E\wedge Y]_r=(E\wedge Y)^{-r}(X)$ from $E_\ast(X);$ for example, we may have available a universal coefficient theorem.

\begin{lemma}
\label{lem:p2c06.2}
We have the following commutative diagram.
$$\begin{tikzcd}
\left[X,Y\right]_\ast && \left[X,E\wedge Y\right]_\ast \\
\\
&\operatorname{Hom}_{\pi_\ast(E)}(E_\ast(X),E_\ast(Y))
\arrow["\alpha", from=1-1, to=3-2]
\arrow["B", from=1-1, to=1-3]
\arrow["p", from=1-3, to=3-2]
\end{tikzcd}$$
Here $\alpha$ is defined by 
$$\alpha(f)=f_\ast:E_\ast(X)\longrightarrow E_\ast(Y),$$
while $p$ is the homomorphism of the universal coefficient theorem, defined by 
$$(p(h))(k)=\left<h,k\right>\in \pi_\ast(E\wedge Y)$$
In the last formula we have $h\in(E\wedge Y)^\ast(X)$, $k\in E_\ast(X)$, and the Kronecker product $\left<h,k\right>$ is defined using the obvious pairing of $E\wedge Y$ and $E$ to $E\wedge Y$.
\end{lemma}

The proof of the lemma from the definitions is easy diagram-chasing. The lemma is of course mainly useful when $p$ is an isomorphism; but since $E\wedge Y$ is a module-spectrum over $E$, Lemma \ref{lem:p2c04.2} shows that $p$ is an isomorphism when $E$ is as in \hyperref[sec:p2c2]{\S 2}, and $X=\mathbb{CP}^\infty, BU, MU,$ etc.

Let $E$ be a ring-spectrum which satisfies the assumptions made in \hyperref[sec:p2c2]{\S 2}. Then we can consider the following two maps.
$$E\simeq E\wedge S^0\longrightarrow E\wedge MU$$
$$MU\simeq S^0\wedge E\Longrightarrow E\wedge MU.$$
Both are of course maps of ring-spectra. The generators $x^E$ and $x^{MU}$ will yield two generators in $(E\wedge MU)^\ast(\mathbb{CP}^\infty)$, and these generators may well be different. In order to remember which is which, we call them $x^E$ and $x^{MU}$ also (abusing notation to avoid complicating it). Our next task is to compare $x^E$ and $x^{MU}$.
\begin{lemma}
\label{lem:p2c06.3}
In $(E\wedge MU)^\ast(\mathbb{CP}^\infty)$ we have 
$$x^{MU}=\sum_{i\geq 0}\left(u^E\right)^{-1}b_i^E\left(x^E\right)^{i+1}.$$
Note that the coefficients $\left(u^E\right)^{-1}b_i^E$ lie in $\pi_\ast(E\wedge MU).$
\end{lemma}
\begin{proof}
Apply Lemma \ref{lem:p2c06.2} to the case $X=\mathbb{CP}^\infty$, $Y=MU$. Since $x^{MU}$ is a reduced class, so is $Bx^{MU}.$ by definition, we have 
$$\left(\alpha x^{MU}\right)\left(u^E\beta_{i+1}^E\right)=b_i^E.$$
But we also have 
$$\left(p\left(x^E\right)^j\right)\left(b_i^E\right)=\begin{cases}1 & (i=j) \\ 0 (i\neq j)\end{cases}$$
The result follows by comparing these formulae, since $p$ is an isomorphism. 
\end{proof}


In order to exploit this result, let $g\left(x^E\right)$ be the formal power-series
\begin{equation}
\label{eqn:p2c06.4}
\tag{6.4}
    g\left(x^E\right)=\sum_{i\geq 0}\left(u^E\right)^{-1}b_i^E\left(x^E\right)^{i+1},
\end{equation}
with coefficients in $\pi_\ast(E\wedge MU),$ and let $g^{-1}$ be the inverse power-series, so that
$$x^E=g^{-1}x^{MU}.$$

\begin{corollary}
\label{cor:p2c06.5}
After applying the homomorphisms
$$\pi_\ast(E)\longrightarrow \pi_\ast(E\wedge MU)$$
$$\pi_\ast(MU)\longrightarrow \pi_\ast(E\wedge MU)$$
the formal products $\mu^E,\mu^{MU}$ are related by 
$$\mu^{MU}\left(x_1^{MU},x_2^{MU}\right)=g\left(\mu^E\left(g^{-1}x_1^{MU},g^{-1}x_2^{MU}\right)\right).$$
The proof is immediate from Lemma \ref{lem:p2c02.13}; or directly, the map $m:\mathbb{CP}^\infty\times\mathbb{CP}^\infty\longrightarrow \mathbb{CP}^\infty$ yields an induced homomorphism $m^\ast$ which commutes with products and limits, so that 
$$m^\ast g\left(x^E\right)=g\left(m^\ast x^E\right).$$
one just rewrites this equation.
\end{corollary}
\begin{corollary}
\label{cor:p2c06.6}
Take $E=H.$ Then after applying the homomorphism $\pi_\ast(MU)\longrightarrow \pi_\ast(H\wedge MU)$ we have 
$$\mu^{MU}(x_1,x_2)=\exp^H(\log^Hx_1+\log^Hx_2),$$
where 
$$\exp^H(x)=\sum_{i\geq 0}b_ix^{i+1},$$
$b_i\in H_{2i}(MU)$ are the usual generators coming from $H_{2i+2}(MU(1))$, and $\log^H$ is the formal power-series inverse to $\exp^H$
\end{corollary}
This is immediate from \eqref{cor:p2c06.5}, using \eqref{ex:p2c02.8}.

This corollary yields a method of calculating the image of $a_{ij}$ in $H_{2(i+j-1)}(MU)$, in terms of the usual base in$H_\ast(MU)$. For example we have 
\begin{align*}
    a_{11} &\longrightarrow 2b_1  \\
    a_{12} &\longrightarrow 3b_2-2b_1^2  \\ 
    a_{13} &\longrightarrow 4b_3-8b_1b_2+4b_1^3  \\
    a_{22} &\longrightarrow 6b_3-6b_1b_2+2b_1^3 \quad \text{etc.} 
\end{align*}
\begin{corollary}
\label{cor:p2c06.7}
Take $E=K$. The after applying the homomorphism $\pi_\ast(MU)\longrightarrow \pi_\ast(K\wedge MU),$ we have
$$\mu^{MU}(x_1,x_2)=g\left(g^{-1}x_1+g^{-1}x_2+\left(g^{-1}x_1\right)\left(g^{-1}x_2\right)\right),$$
where 
$$g(x)=\sum_{i\geq 0} u^{-1}b_ix^{i+1},$$
$u\in \pi_2(K)$, $b_i\in K_0(MU)$ are the generators defined above and $g^{-1}$ is the formal power-series inverse to $g$.
\end{corollary}

This is immediate from \eqref{cor:p2c06.5}, using \eqref{ex:p2c02.8}.

This corollary yields a method of calculating the image of $a_{ij}$ in $K_{2(i+j-1)}(MU)$, in terms of the base in $K_\ast(MU)$. For example, we have 
\begin{align*}
    a_{11} &\longrightarrow u(1+2b_1)  \\
    a_{12} &\longrightarrow u^2(b_1+3b_2-2b_1^2)  \\ 
    a_{13} &\longrightarrow u^3(2b_2-2b_1^1+4b_3-8b_1b_2+4b_1^3)  \\
    a_{22} &\longrightarrow u^3(b_1+6b_2-3b_1^2+6b_3-6b_1b_2+2b_1^3) \quad \text{etc.} 
\end{align*}

We can also use the same method to calculate the Hurewicz homomorphism
$$\pi_\ast(MU)\longrightarrow MU_\ast(MU).$$
For this purpose we need to distinguish between the two copies of $MU$.\\
We borrow the notation of \cite{adams3}, and write
$$\eta_L,\eta_R:\pi_\ast(MU)\longrightarrow MU_\ast(MU)$$
for the homomorphisms induced by the maps
$$MU\simeq MU\wedge S^0\overset{1\wedge i}{\longrightarrow} MU\wedge MU$$
$$MU\simeq S^0\wedge MU\overset{i\wedge 1}{\longrightarrow} MU\wedge MU.$$
The Hurewicz homomorphism is $\eta_R$. The usual action of $\pi_\ast(MU)$ on $MU_\ast(X)$, which works for any $X$, is given for $X=MU$ by $\eta_L$.
\begin{corollary}
\label{cor:p2c06.8}
The value of $\eta_R$ on the generators $a_{ij}$ is given by 
$$\mu^R(x_1,x_2)=g\mu^L\left(g^{-1}x_1,g^{-1}x_2\right)$$
where 
$$\mu^R(x_1,x_2)=\sum_{i,j}(\eta_Ra_{ij})x_1^ix_2^j,$$
$$\mu^L(x_1,x_2)=\sum_{i,j}(\eta_La_{ij})x_1^ix_2^j,$$
$$g(x)=\sum_{i\geq 0}b_i^{MU}x^{i+1},$$
$b_i^{MU}\in MU_{2i}(MU)$ is the generator described in \hyperref[sec:p2c4]{\S 4}, and $g^{-1}$ is the power-series inverse to $g$.
\end{corollary}

This corollary is strictly on the same footing as the preceding two.

This yields a method of calculating $\eta_R(a_{ij})$. For example, we find
\begin{align*}
    \eta_R(a_{11})&=2b_1+a_{11} \\ 
    \eta_R(a_{12})&=(2b_2-2b_1^2)+a_{11}b_1 + a_{12 } \\ 
    \eta_R(a_{13})&=(4b_3-8b_1b_2+4b_1^3)+a_{11}(2b_2-2b_1^2) + a_{13} \\ 
    \eta_R(a_{22})&=(6b_3-6b_1b_2+2b_1^3)+a_{11}(6b_2-3b_1^2) + (2a_{11}+a_{11}^2)b_1+a_{22}. 
\end{align*}
From these formulae for the images of the $a_{ij}$ under the Hurewicz homomorphism
$$\pi_\ast(MU)\longrightarrow MU_{ast}(MU)$$
one can of course deduce the formulae for the images of the $a_{ij}$ under the Hurewicz homomorphisms
$$\pi_\ast(MU)\longrightarrow H_\ast(MU)$$
$$\pi_\ast(MU)\longrightarrow K_\ast(MU).$$
One just applies the maps $MU\longrightarrow H$, $MU\longrightarrow K$. In fact, the map $MU\longrightarrow H$ sends $b_i^{MU}$ to $b_i^H$, and sends $a_{ij}$ to $0$ for $i\geq 1$, $j\geq 1$. The map $MU\longrightarrow K$ sends $b_i^{MU}$ to $u^ib_i^K$, $a_{11}$ to $u$, and $a_{ij}$ to $0$ if $i>1$ or $j>1$.
\end{document}
