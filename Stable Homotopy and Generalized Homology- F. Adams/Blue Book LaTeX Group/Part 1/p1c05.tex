\documentclass[../main]{subfiles}
\begin{document}
\label{sec:p1c5}

\chapter{The Novikov Operations}
The basic analogy which Novikov follows is now: as the Steenrod squares are to the Stiefel-Whitney classes, so the Novikov operations are to the Conner-Floyd characteristic classes. This will be made precise in Theorem \ref{thm:p1c05.1}(vii) below.

\begin{theorem}[S.P. Novikov]
\label{thm:p1c05.1}
For each $\alpha = (\alpha_1, \alpha_2, \alpha_3, \ldots)$ there exists an operation $$s_\alpha : \Omega_U^q(X, Y) \lar{} \Omega_U^{q + 2|\alpha|} (X, Y)$$ with the following properties:

\begin{enumerate}
	\item[(i)] $s_0 = 1$, the identity operation.
	\item[(ii)] $s_\alpha$ is natural: $s_\alpha f^\ast = f^\ast s_\alpha$.
	\item[(iii)] $s_\alpha$ is stable: $s_\alpha \delta = \delta s_\alpha$.
	\item[(iv)] $s_\alpha$ is additive: $s_\alpha (x + y) = (s_\alpha x) + (s_\alpha y).$
	\item[(v)] Cartan formula: $$s_\alpha(xy) = \sum_{\beta + \gamma = \alpha} (s_\beta x) (s_\gamma y).$$ 
	\item[(vi)] Suppose that an element $\omega \in \Omega^2(X)$ is represented by a map $X \lar{g} \mathrm{MU}(1)$. Then $$s_\alpha(\omega) = \sum_i (c_\alpha, b_i) \omega^{i + 1}.$$
	\item[(vii)] Suppose that $\xi$ is an $U(n)$-bundle over $X$, and consider the following diagram.
\end{enumerate}

\begin{center}
\begin{tikzcd}
{\Omega_U^{2n}(E, E_0)} \arrow[rr, "s_\alpha"] &  & {\Omega^{2n + 2|\alpha|}(E, E_0)}\\ &  & \\
\Omega_U^0(X) \arrow[uu, "\phi", "\cong"] &  & 
\Omega_U^{2|\alpha|}(X) \arrow[uu, "\phi", "\cong"]
\end{tikzcd}
\end{center}

(Here the pair $E, E_0$ is as in \hyperref[sec:p1c2]{\S 2}, and $\phi$ is the Thom isomorphism for $\Omega_U^\ast$.) Then we have $$\mathrm{cf}_\alpha(\xi) = \phi^{-1} s_\alpha \phi 1.$$
\end{theorem}

\emph{Explanations:} In (v), the addition of the sequences $\beta$ and $\gamma$ is done term-by-term. The cup product $xy$ may be taken in any one of the three senses explained above, and then the cup product $(s_\beta x)(s_\gamma y)$ is to be taken in the same sense.

For the coefficients $(c_\alpha, b_i)$ in $(vi)$, see the note on Theorem \ref{thm:p1ch04.1}(iv).

\begin{proof}[Sketch Proof]
We take (vii) as our guide. We have a Thom isomorphism $$\phi : \Omega_U^\ast(\mathrm{BU}(n)) \lar{} \tilde \Omega_U^\ast(\mathrm{MU}(n)).$$ Consider the elements $\phi \mathrm{cf}_\alpha \in \tilde \Omega_U^{2n + 2|\alpha|}(\mathrm{MU}(n)).$ They yield a unique element $s_\alpha \in \Omega^{2|\alpha|}(\mathrm{MU})$ (the $\mathrm{Lim}^1$ argument again). This element defines an operation on the cohomology theory $\Omega_U^\ast.$

Property (vii) results immediately from the definition, and properties (ii), (iii) and (iv) are trivial. For example if $x, y : X \lar{} \mathrm{MU}$ are maps, and if we represent $s_\alpha$ by a map $s : \mathrm{MU} \lar{} S^{2a}\mathrm{MU}$, then the maps $s(x + y)$ and $(sx)+(sy) : X \lar{} S^{2a} \mathrm{MU}$ are homotopic, since we are working in a stable category.

Properties (i), (v) and (vi) are deduced from the corresponding properties (i), (iii) and (iv) of the Conner-Floyd classes (Theorem \ref{thm:p1ch04.1}) by using appropriate properties of the Thom isomorphism $\phi$. For example: in proving (v), it is sufficient to consider the case in which $x$ and $y$  are both the identity map $i : \mathrm{MU} \lar{} \mathrm{MU}$ so that $x y$ is the product map $\mu : \mathrm{MU} \wedge \mathrm{MU} \lar{} \mathrm{MU}$. Using the $\mathrm{Lim}^1$ argument again, it is sufficient to consider the case in which $x$ and $y$ are generators for ${\tilde \Omega}_U^{2n}(\mathrm{MU}(n))$, ${\tilde \Omega}_U^{2m}(\mathrm{MU}(m))$. Now we use the fact that if $\xi$ is a $U(n)$ bundle over $X$ and $\eta$ is a $U(m)$-bundle over $Y$ the following diagram is commutative. 

\begin{center}
\begin{tikzcd}
{\tilde \Omega}_U^{p + 2n}(M(\xi)) \otimes {\breve \Omega}^{q + 2m}(\mathrm{MU}(\eta)) \arrow[rr, "\mathrm{product}"] &  & {\tilde \Omega}^{p + q + 2n + 2m}(M(\xi) \wedge M(\eta))                             \\
                                                                                                                      &  & {\tilde \Omega}^{p + q + 2n + 2m}(M(\xi \times \eta)) \arrow[u, equals] \\
\Omega_U^p(X) \otimes \Omega_U^q(Y) \arrow[rr, "\mathrm{product}"] \arrow[uu, "\phi_\xi \otimes \phi_\eta"]           &  & \Omega_U^{p + q}(X \times Y) \arrow[u, "\phi_{\xi \times \eta}"']                   
\end{tikzcd}
\end{center}

The application, of course, is with $\xi$ the universal bundle over $\mathrm{BU}(\eta)$ and $\eta$ the universal bundle over $\mathrm{BU}(m)$.

For (vi) we need to know that for the universal $U(1)$-bundle over $\mathrm{BU}(1)$, the homomorphism $$\Omega_U^{2i}(\mathrm{BU}(1)) \lar{} {\tilde \Omega}_U^{2 i + 2}(\mathrm{MU}(1)) = \Omega_U^{2i + 2}(\mathrm{MU}(1)) \quad i \ge 0$$ carries $\overline \omega^i$ to $\overline \omega^{i + 1}$. (Here $\overline \omega$ is the universal element in $\Omega_U^2(\mathrm{BU}(1))$ or $\Omega_U^2(\mathrm{MU}(1))$.)

Since $s_\alpha$ is a homotopy class of maps $$\mathrm{MU} \lar{} S^{2|\alpha|} \mathrm{MU},$$ it induces a homomorphism $$s_\alpha : H_q(\mathrm{MU}) \lar{} H_{q - 2|\alpha|} (\mathrm {MU}).$$ It is reasonable to ask for this homomorphism to be made explicit. Since we have seen in \hyperref[sec:p1c3]{\S 3} that $H_\ast(\mathrm{MU})$ is a polynomial ring, it is reasonable to ask (i) how $s_\alpha$ acts on products, and (ii) how $s_\alpha$ acts on the generators $b_i'$. Set $\displaystyle b' = \sum_{i = 0}^\infty b_i'$; then it is sufficient to know $s_\alpha(b')$, since one can separate the components again.
\end{proof}

\begin{theorem}\label{thm:p1c05.2}
(i) If $x ,y \in H_\ast(\mathrm{MU})$, then $$s_\alpha (xy) = \sum_{\beta + \gamma = \alpha} (s_\beta x) (s_\gamma y).$$ 
(ii) $\displaystyle s_\alpha(b') = \sum_{i \ge 0} (c_\alpha, b_i)(b')^{i + 1}.$
\end{theorem}

\begin{proof}[Sketch Proof]
Part (i). By Theorem \ref{thm:p1c05.1}(v), we have the commutative diagram. 

\begin{center}
\begin{tikzcd}
\mathrm{MU} \wedge \mathrm{MU} \arrow[rr, "\mu"] \arrow[dd, "\sum_{\beta + \gamma = \alpha} s_\beta \wedge s_\gamma"'] &  & \mathrm{MU} \arrow[dd, "s_\alpha"] \\
&  & \\
\bigvee_{\beta + \gamma = \alpha} S^{2|\beta|} \mathrm{MU} \wedge S^{2|\gamma|} \mathrm{MU} \arrow[rr, "\mu"] &  & S^{2|\alpha|} \mathrm{MU}         
\end{tikzcd}
\end{center}

Pass to induced maps of homology.

Part (ii). Since the generators $b_t'$ come from $\mathrm{MU}(1)$, we can make use of Theorem \ref{thm:p1c05.1}(vi). If $\omega$ is the canonical element of $\Omega^2(\mathrm{MU}(1))$, we wish to compute the effect on homology of the element $\omega^{i + 1} \in \Omega^{2i + 2} (\mathrm{MU}(1))$, that is, the effect of the following composite map.

\begin{center}
\begin{tikzcd}
\mathrm{MU}(1) \arrow[rr, "\Delta"] &  & \mathrm{MU}(1) \wedge \mathrm{MU}(1) \wedge \ldots \mathrm{MU}(1) \arrow[d, "\mu"] & (i + 1) \text { factors} \\
                                    &  & \mathrm{MU}(i + 1)                                                                 &                         
\end{tikzcd}
\end{center} 

Now, the diagonal map $$\mathrm{BU}(1) \lar{\Delta} \mathrm{BU}(1) \times \mathrm{BU}(1) \times \ldots \times \mathrm{BU}(1)$$ induces a map of cohomology given by $$\Delta^\ast (x^{u_1} \otimes x^{u_2} \otimes \ldots \otimes x^{u_{i + 1}}) = x^{u_1 + u_2 + \ldots + u_{i + 1}};$$ therefore it induces a map of homology given by $$\Delta_\ast b_t = \sum_{u_1 + u_2 + \ldots + u_{i + 1} = t} b_{u_1} \otimes b_{u_2} \otimes \ldots \otimes b_{u_{i + 1}}.$$

The map of ${\tilde H}_\ast$ induced by $$\mathrm{BU}(1) \lar{\Delta} \mathrm{BU}(1) \wedge \mathrm{BU}(1) \wedge \ldots \mathrm{BU}(1)$$ is given by the same formula, provided we now interpret $b_0$ as $0$. Next recall that $b_t'$ in $\mathrm{MU}(1)$ corresponds to $b_{t + 1}$ in $\mathrm{BU}(1)$. We deduce that $$\Delta_\ast b_t' = \sum_{u_1 + u_2 + \ldots + u_{i + 1} = t - i} b'_{u_1} \otimes b'_{u_2} \otimes \ldots \otimes b'_{u_{i + 1}}$$ and $$\mu_\ast \Delta_\ast b_t' = \sum_{u_1 + u_2 + \ldots + u_{i + 1} = t - i} b'_{u_1} b'_{u_2} \ldots b'_{u_{i + 1}}.$$ Adding, we see that $$\mu_\ast \Delta_\ast b' = (b')^{i + 1}.$$

By Theorem \ref{thm:p1c05.1}(vi), we have the following commutative diagram.

\begin{center}
\begin{tikzcd}
& S^2 \mathrm{MU} \arrow[rd, "s_\alpha"] & \\
\mathrm{MU}(1) \arrow[ru, "\overline \omega"] \arrow[rr, "{(c_\alpha, b_{|\alpha|}) \overline \omega^{i + 1}}"] & 
& S^{2|\alpha| + 2} \mathrm{MU}
\end{tikzcd}
\end{center}

Pass to induced maps of homology.
\end{proof} 

\begin{corollary}
\label{cor:p1c05.3}
$s_\alpha : H^0(\mathrm{MU}) \lar{} H^{2|\alpha|} (\mathrm{MU})$ is given by $$s_\alpha \phi 1 = \phi c_\alpha.$$
\end{corollary}

\begin{proof}
By Theorem \ref{thm:p1c05.1}(ii), $$s_\alpha(b_i') = \begin{cases}0 & i < |\alpha| \text { (trivially)} \\ (c_\alpha, b_i) 1 & i = |\alpha|.\end{cases}$$ Using Theorem \ref{thm:p1c05.1}(i) we have $$s_\alpha (b_{i_1}' b_{i_2}' \ldots b_{i_r}') = \sum_{\beta_1 + \beta_2 + \ldots + \beta_r = \alpha} (s_{\beta_1} b_{i_1}') (s_{\beta_2} b'_{i_2}) \ldots (s_{\beta_r} b'_{i_r}).$$

If we assume that $i_1 + i_2 + \ldots + i_r = |\alpha|$, then the only terms which can contribute to this sum are those with $$|\beta_1| = i_1, \quad |\beta_2| = i_2, \ldots, |\beta_r| = i_r,$$ and we obtain $$\sum (c_{\beta_1}, b_{i_1}) (c_{\beta_2}, b_{i_2}) \ldots (c_{\beta_r}, b_{i_r}) 1$$ where the sum runs over each such $\beta_1, \beta_2, \ldots, \beta_r$. This of course yields $$(c_\alpha, b_{i_1}, b_{i_2}, \ldots, b_{i_r}) 1.$$ We have shown that $$s_\alpha(\phi x) = (c_\alpha, x) 1$$ for $x \in H_{2|\alpha|} (\mathrm{BU})$. Transposing to cohomology, we obtain $$s_\alpha \phi 1 = \phi c_\alpha.$$
\end{proof}
\end{document}