\documentclass[../main]{subfiles}
\begin{document}
\label{sec:p2c9}
\chapter{Corollaries}
In this section we will record various results which follow from the results in \hyperref[sec:p2c8]{\S 8}, or supplement them, and are needed later.

Recall that the complex manifold $\mathbb {CP}^n$ defines an element $[\mathbb {CP}^n] \in \pi_{2n} (\mathrm {MU})$.

\begin{lemma}
\label{lem:p2c09.1}
With the notation of \hyperref[sec:p2c7]{\S 7}, the image of $[\mathbb {CP}^n]$ in $H_{2n}(\mathrm {MU})$ is $(n + 1)m_n$.
\end{lemma}

\begin{proof}
Algebraic topologists will instantly recognise the formula $$\left(\sum_{i = 0}^\infty b_i\right)_n^{-n - 1}$$ of \eqref{prop:p2c07.5} as giving the normal Chern numbers of $\mathbb {CP}^n$.
\end{proof}
We know from \hyperref[sec:p2c8]{\S 8} that the map $$\pi_\ast(\mathrm{MU}) \lar{} \pi_\ast(\mathrm {MU}) \otimes \bbQ$$ is an injection, and we may identify $\pi_\ast(\mathrm{MU}) \otimes \bbQ$ and $H_\ast(\mathrm{MU}) \otimes \bbQ$. It is often convenient to work in $\pi_\ast(\mathrm{MU}) \otimes \bbQ$, and we now know that we lose nothing in doing so. In what follows, then, we will regard $m_n = m_n^H \in H_{2n}(\mathrm{MU})$ as the element $\tfrac {[\mathbb {CP}^n]} {n + 1}$ of $\pi_\ast(\mathrm {MU}) \otimes \bbQ$. If we do so, we have the following result. 

\begin{corollary}[\plscite{Miščenko [13, Appendix 1]}]
\label{cor:p2c09.2}
The logarithmic series for the formal product $\mu^{\mathrm{MU}}$ may be written $$\log^{H} x^{\mathrm{MU}} = \sum_{n \ge 0} \frac {[\mathbb {CP}^n]} {n + 1} (x^{\mathrm {MU}})^{n + 1}.$$
\end{corollary}

\begin{lemma}
\label{lem:p2c09.3}
Suppose that $R \subset S$ are subrings of the rationals. Then a map $$f : \mathrm{MUR} \lar{} \mathrm{MUS}$$ is determined up to homotopy by $$f_\ast : \pi_\ast(\mathrm{MUR}) \lar{} \pi_\ast(\mathrm{MUS}).$$
\end{lemma}

\begin{proof}
There are many variants possible; we argue as follows. Applying \eqref{lem:p2c04.1} as in the proof of \eqref{thm:p2c08.1}, we see that $f$ is determined up to homotopy by $$f_\ast : \mathrm{MUR}_\ast(\mathrm{MUR}) \lar{} \mathrm{MUR}_\ast(\mathrm {MUS}_\ast).$$ Since $\pi_\ast(\mathrm{MU})$ is torsion-free by [...3], we see that the vertical arrows of the following commutative diagram are monomorphisms.

%TODO: bottom two arrows to the left and the right need a rotated \simeq
\begin{center}
\begin{tikzcd}
\mathrm{MUR}_\ast(\mathrm{MUR}) \arrow[rr, "f_\ast"] \arrow[d]                      &  & \mathrm{MUR}_\ast(\mathrm{MUS}) \arrow[d]                   \\
\mathrm{MUR}_\ast(\mathrm{MUR}) \otimes \bbQ \arrow[rr] \arrow[d]           &  & \mathrm{MUR}_\ast(\mathrm{MUS}) \otimes \bbQ \arrow[d] \\
\pi_\ast(\mathrm {MUQ}) \otimes \pi_\ast(\mathrm {MUR}) \arrow[rr, "1 \otimes f_\ast"] &  & \pi_\ast(\mathrm {MUQ}) \otimes \pi_\ast(\mathrm{MUS})        
\end{tikzcd}
\end{center}
\end{proof}

Next we go back to the work of \eqref{cor:p2c06.8}. We now know that the Hurewicz homomorphism $$\eta_R : \pi_\ast (\mathrm{MU}) \lar{} \mathrm {MU}_\ast (\mathrm {MU})$$ is adequately described by giving $$\eta_R \otimes 1 : \pi_\ast(\mathrm {MU}) \otimes \bbQ \lar{} \mathrm{MU}_\ast(\mathrm {MU}) \otimes \bbQ,$$ and this can be done by giving its effect on the generators $m_i = m_i^H \in \pi_{2i} (\mathrm{MU}) \otimes \bbQ$. For this purpose we propose the following formula. We write $M_j$ for the generator $m_j^{\mathrm{MU}} \in \mathrm{MU}_{2j} (\mathrm {MU})$, to distinguish it from $m_j = m_j^H$.

\begin{proposition}
\label{prop:p2c09.4}
$$\sum_{i \ge 0} (\eta_R m_i)x^{i + 1} = \sum_{i \ge 0} m_i \left(\sum_{j \ge 0} M_j x^{j + 1}\right)^{i + 1}$$
\end{proposition}

\begin{proof}
Consider again the two maps $$\eta_L : \mathrm{MU} \simeq \mathrm{MU} \wedge S^0 \lar{1 \wedge i} \mathrm{MU} \wedge \mathrm{MU}$$ 
$$\eta_L : \mathrm{MU} \simeq S^0 \wedge \mathrm{MU} \lar{i \wedge 1} \mathrm{MU} \wedge \mathrm{MU}$$ 
of \eqref{cor:p2c06.8}. Applying them to $x^{\mathrm{MU}}$, we obtain the two generators in $(\mathrm{MU} \wedge \mathrm {MU})^2 (\mathbb {CP}^\infty)$; we call these generators $x^L$ and $x^R$. (We no longer need $L$ for the Lazard ring) Applying Lemma \ref{lem:p2c06.3}, we find

\begin{equation}
\label{eqn:p2c09.5}
\tag{9.5}
x^R = \sum_{i \ge 0} b_i^{\mathrm{MU}} (x^L)^{i + 1}.
\end{equation}

Passing to the inverse power-series, we find

\begin{equation}
\label{eqn:p2c09.6}
\tag{9.6}
x^H = \sum_{i \ge 0} m_i^{\mathrm{MU}} (x^R)^{i + 1} = \sum_{j \ge 0} M_j (x^R)^{j + 1}
\end{equation}

Now our log series are
$$x^H = \log^L x^L = \sum_{i \ge 0} (\eta_L m_i) (x^L)^{i + 1}$$
$$x^H = \log^R x^R = \sum_{i \ge 0} (\eta_R m_i) (x^R)^{i + 1}$$

So we obtain $$\sum_{i \ge 0} (\eta_R m_i) (x^R)^{i + 1} = \sum_{i \ge 0} (\eta_L m_i) \left(\sum_{j \ge 0} M_j (x^R)^{j + 1}\right)^{i + 1}.$$
This proves the proposition.
\end{proof}
\end{document}