\documentclass[../main]{subfiles}

% Added this too
\renewcommand{\labelenumi}{(\roman{enumi})}
%%%%

\begin{document}
\label{sec:p2c15}
\chapter{Quillen's Idempotent Cohomology Operations}
Suppose given a spectrum $E$ and an abelian group $G$. It may happen that when we form the spectrum $EG$, as in \hyperref[sec:p2c2]{\S 2}, it splits as a sum or product. Examples are given in \cite[Lecture 4]{adams3}. In such cases, it is highly desirable to have a splitting which is canonical and doesn't depend on any choices, I have developed this point in \cite[Lecture 4]{adams3}. In particular, I have made the rather obvious point that one should look for canonical idempotent cohomology operations, that is, idempotent maps
$\varepsilon:EG\longrightarrow EG$.

An important special case is that in which $E = \mathrm{MU}$ and $G=\mathbb{Q}_p$, the integers localized at $p$ (that is, the ring of rational numbers $n/m$ with $m$ prime to $p$.) In this case the possibility of splitting $\mathrm{MU}\mathbb{Q}_p$ was proved by Brown and Peterson \cite{brownpeterson}, and again by Novikov \cite{novikov}; but both methods involved choice.

Quillien has succeeded in giving canonical idempotents $\varepsilon:\mathrm{MU}\mathbb{Q}_p\longrightarrow \mathrm{MU}\mathbb{Q}_p$ (one for each $p$). This is profitable in two ways. Firstly, it means that we no longer have to construct the Brown-Peterson spectrum by synthesis, building it up from its homotopy groups and $k$-invariants; we can construct it by taking $\mathrm{MU}\mathbb{Q}_p$ and splitting off the piece we want. Secondly, we obtain a very precise hold on the Brown-Peterson spectrum, and can obtain information about it by passing to the quotient from $\mathrm{MU}\mathbb{Q}_p$. This process yields good, explicit formulae.

\begin{theorem}
\label{thm:p2c15.1}
Let $d>1$ be an integer, and let $R\subset\mathbb{Q}$ be a subring of the rationals containing $d^{-1}$. Then there is a unique map of ring-spectra $$e=e_d:\mathrm{MU}R\to \mathrm{MU}R$$ satisfying the following conditions.
\begin{enumerate}
    \item $e$ is idempotent: $e^2=e$.
    \item $e$ has the following effect on $\pi_\ast(\mathrm{MU}R)$. 
    $$e[\mathbb{CP}^n]=
        \begin{cases}
            0 & n\equiv-1\pmod{d}\\
            [\mathbb{CP}^n] & n\not\equiv-1\pmod{d}
        \end{cases}$$
\end{enumerate}
Two such idempotents $e_d,e_d$ commute
\end{theorem}

\begin{theorem}[BD, Quillen \cite{quillen}]
\label{thm:p2c15.2}
Let $p$ be a prime. Then there is a unique map of ring-spectra $$\varepsilon=\varepsilon_p:\mathrm{MU}\mathbb{Q}_p\to\mathrm{MU}\mathbb{Q}_p$$ satisfying the following conditions.
\begin{enumerate}
    \item $\varepsilon$ is idempotent: $\varepsilon^2=\varepsilon$.
    \item $\varepsilon$ has the following effect on $\pi_\ast(\mathrm{MU}\mathbb{Q}_p)$. 
    $$\varepsilon[\mathbb{CP}^n]=
        \begin{cases}
            0 & n=p^f-1\text{ for some }f\\
            [\mathbb{CP}^n] & \mathrm{else}
        \end{cases}$$
\end{enumerate}
\end{theorem}  

\begin{proof}[Proof of \eqref{thm:p2c15.2} from \eqref{thm:p2c15.1}]
Take $$\varepsilon=\prod_qe_q$$ where the product ranges over all primes $q\neq p$, observing that the product is convergent in the filtration topology on $\mathrm{MU}\mathbb{Q}_p^\ast(\mathrm{MU}\mathbb{Q}_p)$, which is complete and Hausdorff.
\end{proof}

We turn to consider the proof of \eqref{thm:p2c15.1}. We know from Lemma \eqref{thm:p2c15.2} that so long as $\pi_\ast(E)\longrightarrow\pi_\ast(E)\otimes R$ is iso (which is certainly true
for $E = \mathrm{MU}R$), maps of ring-spectra $g:\mathrm{MU}R\longrightarrow E$ are in $(l-1)$ correspondence with power-series $$g_\ast(x^{\mathrm{MU}})=f(e^\mathrm{E})=\sum_{i\geq0}d_i(x^{\mathrm{E}})^{i+1}$$ with $u^{\mathrm{E}}d_0=1$, $d_i\in\pi_\ast(E)$. Assume for simplicity that $u^{\mathrm{E}} = 1$, which is the case in the applications. All we have to do is pick the right power-series. Let us consider how the choice of $f$ will affect $$g_\ast:\pi_\ast(\mathrm{MU}R)\longrightarrow\pi_\ast(E).$$

Let us take the primitive elements
$$\log^{\mathrm{MU}}x^{\mathrm{MU}}=\sum_{i\geq0}m_i(x^{\mathrm{MU}})^{i+1}, m_i=\frac{[\mathbb{CP}^i]}{i+1}\in\pi_\ast(\mathrm{MU})\otimes\mathbb{Q}$$
$$\log^{\mathrm{E}}x^{\mathrm{E}}=\sum_{i\geq0}n_i(x^{\mathrm{E}})^{i+1},\text{ say }n_i\in\pi_\ast(E)\otimes\mathbb{Q},n_0=1.$$

Let us define the $\log^{\mathrm{MU}}$ series by $$\mathrm{mog}x^{\mathrm{MU}}=\sum_{i\geq0}(g_\ast m_i)(x^{\mathrm{MU}})^{i+1},$$ so that it serves to store the coefficients $g_\ast m_i$. Let $\exp^{\mathrm{E}}$ be the series inverse to $\log^{\mathrm{E}}$

\begin{proposition}
\label{thm:p2c15.3}
The The elements $g_\ast m_i\in\pi_\ast(E)\otimes\mathbb{Q}$ are given by $$\mathrm{mog}$$ or equivalently $$\mathrm{mog}z=\log^{\mathrm{E}}(f^{-1}z).$$
\end{proposition}
For our applications we need to know how to construct $f$ given the coefficients $g_\ast m_i$, and the appropriate formula is as follows.
\begin{corollary}
\label{cor:p2c15.4}
$f^{-1}z=\exp^{\mathrm{E}}\mathrm{mog}z.$
\end{corollary} 
\begin{proof}[Proof of (\ref{thm:p2c15.3}), (\ref{cor:p2c15.4})]
The element $$\log^{\mathrm{MU}}x^{\mathrm{MU}}=\sum_{i\geq0}m_i(x^{\mathrm{MU}})^{i+1}$$ is primitive. Therefore $$g_\ast \log^{\mathrm{MU}}x^{\mathrm{MU}}=\sum_{i\geq0}(g_\ast m_i)(fx^P\mathrm{E})^{i+1}=\mathrm{mog}(fx^{\mathrm{E}})$$ is primitive. But the primitive elements in $\tilde{\mathrm{E}\mathbb{Q}}^\ast(\mathbb{CP}^\infty)$ form a free module over $\pi_\ast(\mathrm{E}\mathbb{Q})$, with one generator $\log^{\mathrm{E}}x^{\mathrm{E}}$; and we check that $\mathrm{mog}(fx^{\mathrm{E}})$ has first term $x^{\mathrm{E}}$; so $$\mathrm{mog}(fx^{\mathrm{E}})=\log^{\mathrm{E}} x^{\mathrm{E}}.$$ This proves \eqref{thm:p2c15.3} and \eqref{cor:p2c15.4}.
\end{proof}

Next suppose given a formal product $\mu$, over a ring $R$, and consider formal power-series, with zero constant term, over $R$. We can make these formal power-series into an abelian group by defining $$\sigma+_\mu\tau=\mu(\sigma,\tau).$$ Subtraction in this abelian group will be written $-_\mu$. If our ring $R$ also contains $d^{-1}$, we can divide by $d$ in this abelian group; we write $$\sigma=\left(\frac1d\right)_\mu\tau$$ for the solution of $$\tau=\sigma+_\mu\sigma+_\mu+\cdots+_\mu\sigma\:\text{ ($d$ summands)}$$

If our ring $R$ contains $\mathbb{Q}$, we can write $$\sigma+_\mu\tau=\exp(\log\sigma+\log\tau)$$ where $\exp$ and $\log$ are as in \hyperref[sec:p2c7]{\S 7}.

\begin{proof}[Proof of \eqref{thm:p2c15.1}]
Our proposal is to take
\begin{equation}
\label{eqn:p2c15.5}
\tag{15.5}
\mathrm{mog}z=\log z -\frac1d(\log\zeta_1z+\log\zeta_2z+\cdots+\log\zeta_dz).
\end{equation}
Here $\zeta_1,\zeta_2,\dots,\zeta_d$ are the complex $d$-th roots of 1, and $$\log z=\sum_{i\geq0}m_iz^{i+1},m_i=\frac{[\mathbb{CP}^i]}{i+1}$$ as it should be for $\mathrm{MU}$ or $\mathrm{MU}R$. It is easy to see that this power-series \eqref{eqn:p2c15.5} has the coefficients $g_\ast(m_i)$ given in \eqref{thm:p2c15.1}(ii). A priori the coefficients of $\mathrm{mog}z$ lie in $\pi_\ast(\mathrm{MU})\otimes\mathbb{Q}[\exp2\pi i/d]$.

Applying $\exp$ to \eqref{eqn:p2c15.5}, we get 
\begin{equation}
\label{eqn:p2c15.6}
\tag{15.6}
f^{-1}z=z_\mu\left(\frac1d\right)_\mu(\zeta_1z+_\mu\zeta_2z+_\mu+\cdots+_\mu\zeta_dz).
\end{equation}
For any $\zeta_1,\zeta_2,\dots,\zeta_d$ we can consider $$\zeta_1z+_\mu\zeta_2z+_\mu+\cdots+_\mu\zeta_dz$$ as a formal power-series with coefficients in $$\pi_\ast(\mathrm{MU})\times\mathbb{Z}[\zeta_1,\zeta_2,\dots,\zeta_d].$$ The coefficients are clearly polynomials symmetric in $\zeta_1, \zeta_2,\dots,\zeta_d$ so we can write them in terms of the elementary symmetric functions $\sigma_1,\sigma_2,\dots,\sigma_d$. When we substitute for $$\zeta_1,\zeta_2,\dots,\zeta_d$$ the complex $d$-th roots of 1, we have $$\sigma_1=0,\dots,\sigma_{d-1}=0,\sigma_d=(-1)^{d-1}.$$ We obtain a power-series with coefficients in $\pi_\ast(\mathrm{MU})$.

So \eqref{eqn:p2c15.6} shows that $f^{-1}z$, and hence $fz$, has coefficients in $\pi_\ast(\mathrm{MU})\otimes\mathbb{Z}\left[\frac1d\right]$. This proves the existence of a map $e:\mathrm{MU}R\longrightarrow\mathrm{MU}R$ of ring-spectra satisfying \eqref{thm:p2c15.1}(ii).

The fact that $e_d$ is idempotent follows from the fact that its effect on $\pi_\ast(\mathrm{MU}R)$ is obviously idempotent, by Lemma \ref{lem:p2c09.3}. The fact that two such idempotents $e_d,d_\delta$ commute is proved in the same way. 
\end{proof}
\end{document}